\begin{table}[ht!]
\def\arraystretch{1.5}
\begin{tabular}{| m{3em} | m{12em} | m{17em} |}
 \hline
 \textbf{TIPO} & \textbf{ENDPOINT} & \textbf{DESCRIZIONE} \\
 \hline \hline
 
 POST & /oauth2/token & Restituisce un token Oauth2 con determinati permessi che possa autenticare le chiamate future. \\ 
 \hline
 
 GET & /surveys & Restituisce tutti i questionari presenti nell'ambiente Qualtrics. \\
 \hline
 
 GET & /survey-definitions\newline / \textbf{\{\{surveyID\}\}} & Restituisce i dati completi relativi al questionario \textbf{\{\{surveyID\}\}} come il numero di domande, i blocchi, meta-dati per la visualizzazione web, le domande e i blocchi \textit{(seppur in maniera poco leggibile)}. \\
 \hline
 
 GET & /survey-definitions\newline /\textbf{\{\{surveyID\}\}}/questions & Restituisce tutte le domande del questionario \textbf{\{\{surveyID\}\}} \textit{(senza distinzione fra blocchi)} sotto forma di lista di oggetti e senza ulteriori meta-dati legati alla survey, rendendo i dati molto più leggibili. Questa è una delle chiamate eseguite solo in fase di test ed è stata utilizzata unicamente per osservare e studiare il formato dei dati delle varie domande tramite Postman. \\
 \hline
 
 GET &
 /survey-definitions\newline /\textbf{\{\{surveyID\}\}}\newline /\textbf{\{\{blockID\}\}} &
 Restituisce i dati relativi al blocco\newline \textbf{\{\{blockID\}\}}, tra i quali anche gli ID delle domande facenti parte di quel blocco. Anche questa chiamata è stata utilizzata unicamente a scopo di test. \\
 \hline

 POST & /survey-definitions\newline /\textbf{\{\{surveyID\}\}}/responses & Invia le risposte al questionario \textbf{\{\{surveyID\}\}} al server. \\
 \hline
 
\end{tabular}
\caption{Elenco riassuntivo chiamate API}
\label{table:api}
\end{table}