\cleardoublepage
\phantomsection
\addcontentsline{toc}{chapter}{Introduzione}
\chapter*{Introduzione}

\textbf{Depressione} e \textbf{ansia} sono i problemi più comuni sperimentati dagli studenti universitari; questo è dovuto al fatto che quello universitario è un periodo decisivo della vita di ogni persona che si
ritrova a dover prendere molte decisioni importanti per il proprio futuro e a doversi adattare ad ambienti nuovi, spesso anche a città nuove.
Tutti questi fattori possono portare a grandi quantità di \textbf{stress} che può determinare la nascita di problematiche maggiori.

È ormai risaputo che questo genere di disturbi è spesso più diffuso tra gli studenti universitari \textit{(circa il 30\% degli studenti universitari soffre di depressione)} \cite{MilanoSFU} e spesso tutto questo sfocia in problematiche ancora più gravi (istinti suicidi, abuso di sostanze, autolesionismo, ...).
È chiaro dunque che esiste un problema legato alla salute mentale degli studenti che va
affrontato il prima possibile per evitare ripercussioni più gravi; molte istituzioni spesso mettono a disposizione del personale con cui trattare questi argomenti ma, purtroppo, non tutti sono disposti a parlare dal vivo dei propri problemi, che sia per vergogna o per mancanza di disponibilità. Non solo, in un periodo come questo appena vissuto, in cui una pandemia ha colpito il mondo intero, è ancor più difficile per chi soffre di questo tipo di problemi potersi mettere in contatto con chi di dovere.

Per questo motivo è utile e necessario avere una via alternativa, che possa essere percorsa anche da chi non ha disponibilità agli incontri dal vivo, all'identificazione e potenziale trattamento di questo genere di patologie. L'idea è utilizzare la tecnologia che usiamo già giorno per giorno per svolgere innumerevoli altri compiti.
Lo scopo di questo progetto è dunque la realizzazione di un'applicazione a supporto della salute mentale che permetta, tramite degli \textbf{\textit{screening}}, di identificare potenziali patologie accusate da uno studente e di affrontare un percorso di guarigione costituito da \textbf{video-lezioni} ed \textbf{esercizi}, composti da questionari in diversa forma, mirati ad aumentare la consapevolezza del paziente sulle sue patologie e su come trattarle giorno per giorno.
Le lezioni e gli esercizi saranno erogati settimanalmente, proprio come se fossero delle vere sedute dallo psicologo, in modo da permettere un trattamento graduale e il tutto sarà gestito in maniera anonima proprio per permettere la partecipazione anche a chi non è disposto a parlare apertamente.
L'applicazione inoltre si occuperà di mantenere attivo il rapporto con l'utente tramite dei semplici questionari quotidiani che verranno notificati ad un orario a scelta arbitraria.

L'applicazione è stata sviluppata utilizzando il framework di Google \textbf{Flutter}, che utilizza il linguaggio di programmazione \textbf{Dart}. La scelta di utilizzare questa tecnologia è supportata dal fatto che Flutter permette di avere un'unica code base da cui poter compilare applicazioni sia per \textbf{iOS} che per \textbf{Android}, i due sistemi operativi leader nel mercato degli smartphone.
Non solo, con la recente pubblicazione di Flutter 2, è possibile utilizzare la stessa code base per compilare anche applicazioni \textbf{web} stabili e applicazioni per Google \textbf{Fuchsia}, il nuovo sistema operativo di Google. Possiamo quindi affermare con sicurezza che la scelta di utilizzare Flutter mira, oltre che alla semplicità di sviluppo per diverse piattaforme, al supporto di tecnologie future.

Nel capitolo \textit{MindBlooming} sono spiegati in dettaglio gli obiettivi dell'applicazione e le funzionalità che essa offre, mentre il capitolo \textit{Progettazione} entra nel dettaglio di come sono effettivamente gestite queste funzionalità e con quali strumenti. Nel capitolo \textit{Implementazione} invece sono descritti i componenti utilizzati per costruire l'applicazione, come sono stati costruiti e come sono stati risolti i problemi incontrati durante lo sviluppo. Infine, nel capitolo \textit{Manuale Utente} è presente una guida all'utilizzo dell'applicazione dal punto di vista degli utenti.