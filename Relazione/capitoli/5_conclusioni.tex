\chapter[Conclusioni e sviluppi futuri]{Conclusioni e\\sviluppi futuri}
L'obiettivo prefissato per questo progetto di stage era lo sviluppo di una versione \textit{alpha} di quello che sarebbe stato \textit{MindBlooming}, una versione che avrebbe stabilito se il progetto fosse stato fattibile, di quali dati avrebbe avuto bisogno, di che struttura ci sarebbe dovuta stare dietro. È innegabile che l'applicazione sia ancora acerba, ma aver potuto stabilire questi punti iniziali permetterà un'evoluzione di essa molto più facilitata sapendo già quali dati creare e/o usare.

\section{Idee future}
Al momento tutto ciò che concerne gli sviluppi futuri è il perfezionamento ed il testing di quanto fatto fino ad ora, in modo da avere una base stabile per poi concentrarsi sull'implementazione di nuove feature come la piantina citata nella \autoref{section:impostazioni} del \autoref{chap:manuale_utente}.
Tra i vari perfezionamenti c'è sicuramente anche un miglioramento di alcune parti dell'UI come ad esempio la Homepage che al momento potrebbe sembrare un po' spoglia e l'aggiunta di immagini significative che accompagnino le patologie in modo da dare un suggerimento immediato di cosa tratta ognuna di esse.

\subsection{Backend proprio}
Un possibile sviluppo futuro è sicuramente lo sviluppo di un backend proprio ideato per l'applicazione stessa che fornirebbe una maggiore flessibilità rispetto a Qualtrics e permetterebbe operazioni come l'aggiunta di nuovi moduli \textit{(per esempio abuso di sostanze, disturbi del comportamento alimentare, disturbi ossessivi compulsivi e così via)}, nuovi screening, nuovi esercizi il tutto senza dover ricompilare l'applicazione poiché sarebbe possibile definire delle strutture dati adeguate a questo dominio e spostare alcune computazioni come quella dell'algoritmo di screening direttamente sul server; questo permetterebbe all'applicazione di offrire sempre nuove funzionalità. Inoltre, per quando riguarda sondaggi e domande la struttura potrebbe rimanere pressoché invariata, mantenendo tutto il lavoro svolto su Qualtrics.

\subsection{E-Coaches}
Un'altra funzionalità aggiuntiva che può essere molto utile agli utenti di questa applicazione è la possibilità di avere un contatto diretto con le tutor che hanno creato i vari moduli o comunque con qualcuno che possa fornire loro consigli utili ed eventualmente indirizzarli verso un trattamento professionale se necessario. Il metodo ideale è quello di utilizzare una chat in tempo reale o comunque un sistema di comunicazione stile e-mail ma che mantenga l'anonimato degli utenti.

\subsection{Integrazione in altri ambienti}
Un altro possibile sviluppo per l'applicazione è l'integrazione di essa in diversi ambienti informatici: questo permetterebbe, per esempio, di poter offrire l'applicazione a diversi atenei e/o aziende che la rendano poi disponibile ai propri studenti/dipendenti, cosa che permetterebbe, oltre a migliorare le condizioni dei diretti interessati, ad offrire a queste istituzioni di avere un feedback diretto di come i loro clienti vivono il loro ambiente, potendo così migliorarlo.
\newpage\null\thispagestyle{empty}\newpage